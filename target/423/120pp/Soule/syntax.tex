\chapter{Syntax}

This chapter discusses lexical rules inferred from the text of ``A
Project-Based Introduction to C++''.  You are to implement a subset
of C++ to include at least the following syntax constructs.

\section{Namespaces}

You are allowed, but not required, to require the declaration
\begin{verbatim}
using namespace std;
\end{verbatim}

The preferred handling would be to require it if system includes are present.

\section{Function prototypes}

120++ allows zero or more parameters which consist of a type followed
by optional square-brackets or ampersand.  Examples:

\begin{verbatim}
type name();
type name(type,type);
type name(type []);
type name(type &);
type name(type *); /* doublecheck */
\end{verbatim}

\section{Function bodies}

120++ functions have a return type, a function name with optional
classname:: prefix, zero or more parameters, declarations and
statements inside appropriate parentheses and curly braces. Examples:

\begin{verbatim}
type name() { declarations statements }
type name::name() { declarations statements }
type name(type name,type name) { declarations statements }
\end{verbatim}

\section{Classes}

Although the Soule text mentions the existence of structs and typedefs
in C, it does not use them and 120++ will not include them.
120++ does include simple classes, with no inheritance.

\begin{verbatim}
class name { fields methods } ;
\end{verbatim}

\section{Variable Declarations}

120++ includes common forms of C++ variable declarations.
Note that variable declarations are all global or at the beginning
of their respective scope blocks, with an allowance for variables
declared spontaneously inside for-loops.  This is a restriction on
standard C++ which allows variable declarations at any point within
any statement block.

\begin{verbatim}
type name;
type name1, name2, name3;
type name4 = value;
const type name4 = value;
type name[size];
type *name;
type *name[size1][size2];
\end{verbatim}

Note that although I saw in one place in the Soule text some code
that suggested the pattern

\begin{verbatim}
type [size] name;
\end{verbatim}

I cannot confirm this as legal C++ and it is not in 120++

\section{Statements}

Statements in 120++ include the following, along with void
function calls.
Statements are often terminated with semi-colons.

\begin{verbatim}
if (expr) statement
if (expr) { statements }
if (expr) { statements } else { statements }
if (expr) { statements } else statement
switch (expr) { statements }
case intvalue: statements break
case charvalue: statements break
case (value): statements break
case val1: case val2: statements break
default : statements
break
do { statements } while (expr);
while (expr) { statements }
for (type var = val ; expr ; expr) { statements }
for (var = val ; expr ; expr) { statements }
\end{verbatim}

\section{Expressions}

\begin{verbatim}
cout << string ;
cout << string1 << expr << string2 ;
cin >> name ;
cin >> name >> name2 >> name3 ;
&x
*p
p->field
NULL
new type
new type(args)
(type)(expr)
delete p
name1 = expr ;
name1.name2 = expr;
return expr;
expr op expr
expr++
expr--
(expr)
f()
f(expr,expr)
type(expr)
name1.name2
\end{verbatim}
